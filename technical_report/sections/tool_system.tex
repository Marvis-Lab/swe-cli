\section{Tool System}
\label{sec:tool_system}

The tool system provides the agent with a comprehensive set of capabilities for interacting with the development environment. This section describes the tool registry architecture, the full tool catalog, MCP and LSP integrations, safety mechanisms, and batch execution.

\subsection{Tool Registry}
\label{sec:tool_registry}

The \texttt{ToolRegistry} (\texttt{core/context\_engineering/tools/registry.py}) implements a dispatch pattern that routes tool calls to specialized handlers. Each handler registers its supported tool names and schemas; the registry aggregates these into a unified interface presented to the LLM.

The registry performs several responsibilities:
\begin{packeditemize}
\item \textbf{Schema aggregation:} Collects tool schemas from all handlers and presents them to the agent.
\item \textbf{Dispatch:} Routes incoming tool calls to the appropriate handler based on tool name.
\item \textbf{Context injection:} Wraps each call with a \texttt{ToolExecutionContext} containing the mode manager, approval manager, undo manager, and task monitor.
\item \textbf{Plan mode blocking:} Checks whether the current mode allows the requested tool. Tools not in the \texttt{PLAN\_READ\_ONLY\_TOOLS} set are blocked in Plan mode with an informative error.
\end{packeditemize}

\subsection{Tool Catalog}
\label{sec:tool_catalog}

\Cref{tab:tools} summarizes the tool categories and their constituent tools.

\begin{table}[t]
\centering
\caption{Tool categories in \name with representative tools and their handlers.}
\label{tab:tools}
\small
\begin{tabular}{@{}llp{4.5cm}@{}}
\toprule
\textbf{Category} & \textbf{Handler} & \textbf{Tools} \\
\midrule
File Operations & \texttt{FileToolHandler} & \texttt{read\_file}, \texttt{write\_file}, \texttt{edit\_file}, \texttt{list\_files}, \texttt{search} \\
\addlinespace
Process Mgmt & \texttt{ProcessToolHandler} & \texttt{run\_command}, \texttt{list\_processes}, \texttt{get\_process\_output}, \texttt{kill\_process} \\
\addlinespace
Web Fetch & \texttt{WebToolHandler} & \texttt{fetch\_url} \\
\addlinespace
Web Search & \texttt{WebSearchHandler} & \texttt{web\_search} \\
\addlinespace
Notebooks & \texttt{NotebookEditHandler} & \texttt{notebook\_edit} \\
\addlinespace
User Input & \texttt{AskUserHandler} & \texttt{ask\_user} \\
\addlinespace
MCP Tools & \texttt{McpToolHandler} & Dynamic (discovered at runtime) \\
\addlinespace
Screenshots & \texttt{ScreenshotToolHandler} & \texttt{capture\_screenshot}, \texttt{capture\_web\_screenshot} \\
\addlinespace
Task Mgmt & \texttt{TodoHandler} & \texttt{write\_todos}, \texttt{update\_todo}, \texttt{complete\_todo}, \texttt{list\_todos} \\
\addlinespace
Thinking & \texttt{ThinkingHandler} & \texttt{think} (visibility control) \\
\addlinespace
Discovery & \texttt{SearchToolsHandler} & \texttt{search\_tools} \\
\addlinespace
Batch & \texttt{BatchToolHandler} & \texttt{batch\_tool} (parallel/serial) \\
\addlinespace
PDF & \texttt{PDFTool} & \texttt{read\_pdf} \\
\addlinespace
Completion & \texttt{TaskCompleteTool} & \texttt{task\_complete} \\
\bottomrule
\end{tabular}
\end{table}

\subsection{Model Context Protocol (MCP) Integration}
\label{sec:mcp}

\name integrates the Model Context Protocol~\cite{anthropic2024mcp} for extensible tool discovery. MCP servers expose additional tools that can be dynamically loaded into the agent's tool set. The integration includes:

\begin{packeditemize}
\item \textbf{Server management:} Users add, enable, and disable MCP servers via CLI commands (\texttt{swecli mcp add/enable/disable}).
\item \textbf{Token-efficient lazy discovery:} Rather than including all MCP tool schemas in every LLM call (which can consume significant context), \name maintains a \texttt{discovered\_mcp\_tools} set. Only tools that have been explicitly searched for or previously used are included in the schema payload. The \texttt{search\_tools} tool allows the agent to discover relevant MCP tools by keyword.
\item \textbf{Dynamic dispatch:} The \texttt{McpToolHandler} forwards tool calls to the appropriate MCP server, handling serialization and error propagation.
\end{packeditemize}

\subsection{Language Server Protocol (LSP) Integration}
\label{sec:lsp}

\name leverages the Language Server Protocol~\cite{microsoft2016lsp} for semantic code operations. The symbol tools (\texttt{find\_symbol}, \texttt{find\_referencing\_symbols}) communicate with language servers to provide:

\begin{packeditemize}
\item Symbol definition lookup across the project
\item Reference finding (all usages of a symbol)
\item Type information and hover documentation
\end{packeditemize}

The system supports 20+ programming languages through their respective language servers, including Python (Pyright/Pylsp), TypeScript (tsserver), Rust (rust-analyzer), Go (gopls), Java (Eclipse JDT), C/C++ (clangd), and many others. Language servers are managed as background processes and communicate via JSON-RPC.

\subsection{Safety Mechanisms}
\label{sec:safety}

The tool system implements multiple safety layers:

\begin{packeditemize}
\item \textbf{Dangerous command blocking:} The \texttt{BashTool} maintains a blocklist of dangerous commands (\eg \texttt{rm -rf /}, \texttt{mkfs}, \texttt{dd if=/dev/zero}) and rejects them outright.
\item \textbf{Approval workflows:} The \texttt{ApprovalManager} (\texttt{core/runtime/approval/manager.py}) gates operations classified as potentially destructive. When triggered, the system presents an approval dialog to the user through the active UI interface.
\item \textbf{Plan mode restrictions:} In Plan mode, the tool registry blocks all write operations, ensuring the planning agent cannot modify the codebase.
\item \textbf{Undo tracking:} The \texttt{UndoManager} records file operations (create, modify, delete) with before/after state, enabling rollback of recent changes. The history is persisted to a JSONL log and capped at 50 operations.
\end{packeditemize}

\subsection{Batch Tool Execution}
\label{sec:batch}

The \texttt{BatchToolHandler} enables the agent to invoke multiple tools in a single turn, reducing round-trip overhead. It supports two execution modes:

\begin{packeditemize}
\item \textbf{Parallel:} Independent tool calls execute concurrently via thread pool.
\item \textbf{Serial:} Dependent tool calls execute sequentially, with each result available to subsequent calls.
\end{packeditemize}

This is particularly useful for operations like reading multiple files simultaneously or running search queries in parallel.
